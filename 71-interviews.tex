\section{Interviews}
\label{appendix:interviews}


\subsection{Interview A}

\begin{code}
\begin{minted}{java}
public class FilenameUtils {
	...
	/**
	* Determines if Windows file system is in use.
	* 
	* @return true if the system is Windows
	*/
	static boolean ...() {
		return SYSTEM_SEPARATOR == WINDOWS_SEPARATOR;
	}
	...
}
\end{minted}
\caption{Commons IO 2.4 FilenameUtils.isSystemWindows Variant 1}
\label{lst:filenameutils-issystemwindows-variant1}
\end{code}

\begin{code}
\begin{minted}{java}
public class FilenameUtils {
	...
	static boolean ...() {
		return SYSTEM_SEPARATOR == WINDOWS_SEPARATOR;
	}
	...
}
\end{minted}
\caption{Commons IO 2.4 FilenameUtils.isSystemWindows Variant 2}
\label{lst:filenameutils-issystemwindows-variant2}
\end{code}

\begin{code}
\begin{minted}{java}
public class IOUtils {
	...
    /**
     * Compare the contents of two Readers to determine if they are equal or
     * not, ignoring EOL characters.
     * <p>
     * This method buffers the input internally using
     * <code>BufferedReader</code> if they are not already buffered.
     *
     * @param input1  the first reader
     * @param input2  the second reader
     * @return true if the content of the readers are equal (ignoring EOL differences), 
     *	false otherwise
     * @throws NullPointerException if either input is null
     * @throws IOException if an I/O error occurs
     * @since 2.2
     */
    public static boolean ...(Reader input1, Reader input2)
            throws IOException {
        BufferedReader br1 = toBufferedReader(input1);
        BufferedReader br2 = toBufferedReader(input2);

        String line1 = br1.readLine();
        String line2 = br2.readLine();
        while (line1 != null && line2 != null && line1.equals(line2)) {
            line1 = br1.readLine();
            line2 = br2.readLine();
        }
        return line1 == null ? line2 == null ? true : false : line1.equals(line2);
    }
    ...
}
\end{minted}
\caption{Commons IO 2.4 IOUtils.contentEqualsIgnoreEOL Variant 1}
\label{lst:ioutils.contentequalsignoreeol-variant1}
\end{code}

\begin{code}
\begin{minted}{java}
public class IOUtils {
	...
    public static boolean ...(Reader input1, Reader input2)
            throws IOException {
        BufferedReader br1 = toBufferedReader(input1);
        BufferedReader br2 = toBufferedReader(input2);

        String line1 = br1.readLine();
        String line2 = br2.readLine();
        while (line1 != null && line2 != null && line1.equals(line2)) {
            line1 = br1.readLine();
            line2 = br2.readLine();
        }
        return line1 == null ? line2 == null ? true : false : line1.equals(line2);
    }
    ...
}
\end{minted}
\caption{Commons IO 2.4 IOUtils.contentEqualsIgnoreEOL Variant 2}
\label{lst:ioutils.contentequalsignoreeol-variant1}
\end{code}

\newpage
\begin{code}
\begin{minted}{java}
public class Monitor {
/**
   * Enters this monitor. Blocks at most the given time.
   *
   * @return whether the monitor was entered
   */
  public boolean _(long time, TimeUnit unit) {
    long timeoutNanos = unit.toNanos(time);
    final ReentrantLock lock = this.lock;
    if (!fair && lock.tryLock()) {
      return true;
    }
    long deadline = System.nanoTime() + timeoutNanos;
    boolean interrupted = Thread.interrupted();
    try {
      while (true) {
        try {
          return lock.tryLock(timeoutNanos, TimeUnit.NANOSECONDS);
        } catch (InterruptedException interrupt) {
          interrupted = true;
          timeoutNanos = deadline - System.nanoTime();
        }
      }
    } finally {
      if (interrupted) {
        Thread.currentThread().interrupt();
      }
    }
  }
}

// Usage example:
monitor._(5, TimeUnit.SECONDS);
try {
	// do things while occupying the monitor
} finally {
	monitor.leave();
}
\end{minted}
\caption{Guava 17.0 Monitor.enter Variant 1}
\label{lst:guava-monitorenter-variant1}
\end{code}

\newpage
\begin{code}
\begin{minted}{java}
public class Monitor {
  public boolean _(long time, TimeUnit unit) {
    long timeoutNanos = unit.toNanos(time);
    final ReentrantLock lock = this.lock;
    if (!fair && lock.tryLock()) {
      return true;
    }
    long deadline = System.nanoTime() + timeoutNanos;
    boolean interrupted = Thread.interrupted();
    try {
      while (true) {
        try {
          return lock.tryLock(timeoutNanos, TimeUnit.NANOSECONDS);
        } catch (InterruptedException interrupt) {
          interrupted = true;
          timeoutNanos = deadline - System.nanoTime();
        }
      }
    } finally {
      if (interrupted) {
        Thread.currentThread().interrupt();
      }
    }
  }
}
\end{minted}
\caption{Guava 17.0 Monitor.enter Variant 2}
\label{lst:guava-monitorenter-variant2}
\end{code}

\subsection{Questions to ask during the interview}

\subsubsection{General questions before the interview}

\begin{enumerate}
\item Do you have practical experience with the Java programming language?
\item If yes, how many years of experience do you have?
\item With which other languages do you have practical experience?
\end{enumerate}

\subsubsection{Questions to ask per code sample}
\begin{enumerate}
\item Do you have experience with \textbf{**domain of code sample**}
\item Do you understand and can you explain what this code does?
\end{enumerate}

\section{Interview script}

\begin{enumerate}
\item Please provide a name for the method with the missing name.

\item When unable 
	\begin{enumerate}
	\item ask: Do you have experience/are familiar with \textbf{**domain of code sample**}
	\item When negative: explain/provide reading material about the domain.
	\item Please provide a name for the method with the missing name.
	\item When unable, ask: Do you understand/can you explain what this code does?
	\item When negative: give more contextual information
	\item Repeat last two steps until all contextual information is provided.
	\end{enumerate}
\item When able
	\begin{enumerate}
	\item ask: Can you explain why you chose this particular name and how did you construct it?
	\item ask: Do you have experience with \textbf{**domain of code sample**}
	\item Do you understand and can you explain what this code does?
	\end{enumerate}
\end{enumerate}