%% Based on a TeXnicCenter-Template by Gyorgy SZEIDL.
%%%%%%%%%%%%%%%%%%%%%%%%%%%%%%%%%%%%%%%%%%%%%%%%%%%%%%%%%%%%%

%------------------------------------------------------------
%
\documentclass{article}%
%Options -- Point size:  10pt (default), 11pt, 12pt
%        -- Paper size:  letterpaper (default), a4paper, a5paper, b5paper
%                        legalpaper, executivepaper
%        -- Orientation  (portrait is the default)
%                        landscape
%        -- Print size:  oneside (default), twoside
%        -- Quality      final(default), draft
%        -- Title page   notitlepage, titlepage(default)
%        -- Columns      onecolumn(default), twocolumn
%        -- Equation numbering (equation numbers on the right is the default)
%                        leqno
%        -- Displayed equations (centered is the default)
%                        fleqn (equations start at the same distance from the right side)
%        -- Open bibliography style (closed is the default)
%                        openbib
% For instance the command
%           \documentclass[a4paper,12pt,leqno]{article}
% ensures that the paper size is a4, the fonts are typeset at the size 12p
% and the equation numbers are on the left side
%
\usepackage{amsmath}%
\usepackage{amsfonts}%
\usepackage{amssymb}%
\usepackage{graphicx}
\usepackage{varwidth} % http://ctan.org/pkg/varwidth
\usepackage{pgfgantt}
\usepackage{graphicx} % http://nl.wikibooks.org/wiki/LaTeX/Figuren_en_drijvende_omgevingen
\usepackage[hidelinks]{hyperref}
\usepackage{minted}
\usepackage{listings}

\setminted[java]{linenos=true,tabsize=4,style=colorful}

\begin{document}

% Title page
\title{Found in practice: Strategies for naming methods}
\author{Jeroen Peeters
\\The University of Amsterdam}
\date{November 4, 2015 \\\vspace*{12.5cm}\small{Die Grenzen meiner Sprache bedeuten die Grenzen meiner Welt\\--- Ludwig Wittgenstein ---}} % The limits of my language mean the limits of my world
\maketitle
\thispagestyle{empty}
\newpage

% Title page2
\begin{center}
Master Thesis \\
Software Engineering \\
\vspace{1cm}

\vspace{1cm}
Supervisor: dr. Alexander Serebrenik\\
\hspace{7 mm}dr. Hans Dekkers\\
%\vspace{16cm}
%Kim, bedankt voor je steun!
\end{center}
\thispagestyle{empty}
\newpage

% Abstract page
%\begin{abstract}
%This is a sample document which shows the most important features of the Standard
%\LaTeX\ Journal Article class.
%\end{abstract}
%\newpage

% Table of Contents page
\renewcommand{\contentsname}{ \begin{center} --- Table of Contents --- \end{center}}
\setcounter{tocdepth}{1}
\tableofcontents
\addtocontents{toc}{~\hfill\textbf{Page}\par}
\newpage

\section{Introduction}

\subsection{What's in a name}

Naming pieces of code is one of the most important tasks of a software developer. Giving a piece of code the correct name is important for understandability {\color{red}[CITATION NEEDED]}. But what is the right name? Providing good quality names is generally believed to be of high importance. Though there may be some guidelines, no exact or mechanical process exists to name pieces of code. This thesis seeks to discover the strategies people use to create names for existing pieces of code written in the Java programming language.

\subsubsection{Names in the Java programming language}


Before we continue we should understand which nameable parts exist the Java programming language.
In the Java programming language the following constructs are nameable.\cite{javaLangSpec}

\begin{list}{-}{}
\item Classes
\item Methods
\item Variables
\item Method arguments
\item Return types
\end{list}

Examples of the above constructs are listed in appendix \ref{appendix:java-nameabe-examples}.

\subsection{What's in a good name?}

For computers to interpret and execute program code there is no need for it to be laid out or indented in a way that is preferable to humans. Just the same the computer doesn't care about nice, well understandable and self-explanatory names. We do this to communicate the intent of the code to other developers or even our future self should we come back to extend the code or fix bugs {\color{red}[CITATION NEEDED]}.

\subsubsection{Name characteristics}
\label{sec:name-characteristics}

\textbf{TEXT NEEDED}.
{\color{red}[CITATION NEEDED]}

\begin{list}{-}{Characteristics of names in programming:}
\item Reveal intentions
\item Length
\item One word per concept
\item Use names from the domain
\end{list}


\subsection{Characteristics in code}

\textbf{TEXT NEEDED}.

Zoals besproken op Skype:

De huidige opzet van mijn scriptie/onderzoek blijkt lastig te voltooien. Jij stelde het volgende voor; ik kom een week naar de UvA om studenten en misschien docenten te interviewen over de vraag hoe mensen komen tot de naamgeving van een methode, gegeven de implementatie en context. Deze kennis kan interessant zijn om code automatisch te labelen bij automatische refactoring. Jij merkte op dat het wellicht interessant is om uit te zoeken welke karakteristieken in de code nu echt belangrijk zijn voor het geven van een naam. Het idee is om code met veel verschillende karakteristieken te geven, een complex vraagstuk. Volgens Kahnemen zal dit leiden tot een substituut vraag, welke karakteristieken zijn nu nog belangrijk? Altijd dezelfde? Of altijd de eerste N die men tegenkomt bij het lezen van de code. Maakt de context uberhaupt uit? of is alleen de implementatie genoeg? of is misschien juist alleen de context genoeg voor het geven van een naam? Worden andere namen gegeven bij afwezigheid van context of implementatie. Verder kan nog worden gekeken naar de strategie die mensen kiezen: beantwoorden ze elke vraag op eenzelfde manier waardoor de gegeven namen ook vergelijkbaar zijn opgebouwd. Mogelijkheid is ook dat mensen geen naam kunnen geven. 

Jij noemde ook inconsistentie als mogelijkheid om naar te kijken en je noemde als voorbeeld het gebruik van naamgevingsconventies; zou je nog kort kunnen toelichten wat je hiermee bedoelde?, ik kon dit uit m'n aantekeningen niet direct meer opmaken :S

Het idee is, volgens mij, om het volgende te organiseren: over 3-4 weken een week op de Uva voor het afnemen van interviews. Ik bereid vragen voor op papier met random vragen uit een codebase van github van meestgebruikte open-source software. Interview is interactief volgens 'thinking aloud protocol'.

Deze week + weekend zal ik dit plan verder uitwerken + voorbeeld vragen voorbereiden die wij maandag 9 november kunnen bespreken. Ik zal dan in de middag, na de lunch (rond 13:00u?), staan bij C302.4.

Hans, hartelijk dank voor je tijd zover en je aanbieding om studenten te interviewen.

Groet,
Jeroen
\section{Research Question}

\begin{itemize}
	\item Question 1\\
	{\small Description...}
	
	\item  Question 2\\
	{\small Description...}
	
	\item Can people reverse-engineer method names?\\
	{\small Given a method implementation and contextual information, can people reconstruct the intent of that method and give it an appropriate name?}
	
		\subitem What is the minimal required information?
\end{itemize}
\section{Verification model}

\subsection{Survey: Naming Java Methods}

Naming pieces of code is one of the most important tasks of a software developer. Giving a piece of code the correct name is important for understandability. But what is a correct name? Providing good quality names is generally believed to be of high importance. Though there may be some guidelines, no exact or mechanical process exists to name pieces of code. In order to understand how developers create method names, I've executed a survey among a group of peers.

\subsubsection{Survey setup}

The survey is executed as an online questionnaire which is filled out individually without direct supervision.
Participants are asked to provide names for shown nameless Java methods. Together with the method implementation limited contextual information is given, such as the name of the containing class and examples of how the method is used.
Because the number of methods that can be named by each participant can vary greatly there's no maximum number of questions. Instead, there's a time limit of thirty minutes per participant. After this time, the questionnaire will stop automatically. At any moment, the participant can pause and resume the questionnaire at a later time. 


\subsubsection{Method corpus}



% Them methods used in this survey are taken from randomly chosen from popular open source projects. From each of these projects 10 methods are randomly chosen.

%\begin{itemize}
% 	\item http://www.ohloh.net/p/tomcat
%	\item http://www.ohloh.net/p/maven2
%	\item http://www.ohloh.net/p/hibernate
%	\item http://www.ohloh.net/p/log4j
%	\item http://www.ohloh.net/p/jetty
%	\item http://www.ohloh.net/p/findbugs
%\end{itemize} 

In order to make sure that the methods included in the survey are a representative sample I've employed the following strategy.
The methods used in the survey are taken from open-source Java projects with a high usage frequency. In other words, which projects are depended upon the most?  See appendix \ref{appendix:frequency-of-java-dependencies} for detailed information on how the list was compiled.

\begin{itemize}
	\item JUnit
	\item Log4J
	\item Commons IO
	\item Guava
	\item Commons-lang
	\item Mockito
\end{itemize}

In order to make sure that the methods taken from these projects are also representative I use the SIG maintainability model \cite{sigmodel}. To make sure that any degree of small \& large and simple \& complex method are selected I use the following two properties:

\begin{itemize}
	\item Complexity per unit\\
	{\small The complexity of source code units influences the system’s changeability and its testability.}
	
	\item Unit size\\
	{\small The size of units influences their analysability and testability and therefore of the system as a whole.}
\end{itemize}

\subsubsection{Results}

\subsection{Deducing the model}
\section{Acknowledgement}

TBD...

\newpage
%\nocite{*} %to include also noncited references
\addcontentsline{toc}{section}{References}
\bibliography{mybib}{}
\bibliographystyle{plain}

\appendix

\section{Examples of names in the Java programming language}
\label{appendix:java-nameabe-examples}

\begin{listing}
\caption{Class names}
\label{lst:example-class-names}
\begin{minted}{java}
public class MyClass extends OtherClass {
	...
}
\end{minted}
\end{listing}

\begin{listing}
\caption{Methods}
\label{lst:example-methods}
\begin{minted}{java}
public class _ {
	public void myMethodName(){
		...
	}
}
\end{minted}
\end{listing}

\begin{listing}
\caption{Variables}
\label{lst:example-variables}
\begin{minted}{java}
public class _ {
	private int myClassVariable;
	
	public void ...(){
		String myLocalVariable = "some value";
	}
}
\end{minted}
\end{listing}

\begin{listing}
\caption{Method arguments}
\label{lst:example-method-arguments}
\begin{minted}{java}
public class _ {
	public void _(int myArgument1, String myArgument2){
		...
	}
}
\end{minted}
\end{listing}

\begin{listing}
\caption{Return type}
\label{lst:example-return-types}
\begin{minted}{java}
public class _ {
	public String _(){
		...
	}
	public int _(){
		...
	}
}
\end{minted}
\end{listing}
\section{Interviews}
\label{appendix:interviews}


\subsection{Interview A}

\begin{code}
\begin{minted}{java}
public class FilenameUtils {
	...
	/**
	* Determines if Windows file system is in use.
	* 
	* @return true if the system is Windows
	*/
	static boolean ...() {
		return SYSTEM_SEPARATOR == WINDOWS_SEPARATOR;
	}
	...
}
\end{minted}
\caption{Commons IO 2.4 FilenameUtils.isSystemWindows Variant 1}
\label{lst:filenameutils-issystemwindows-variant1}
\end{code}

\begin{code}
\begin{minted}{java}
public class FilenameUtils {
	...
	static boolean ...() {
		return SYSTEM_SEPARATOR == WINDOWS_SEPARATOR;
	}
	...
}
\end{minted}
\caption{Commons IO 2.4 FilenameUtils.isSystemWindows Variant 2}
\label{lst:filenameutils-issystemwindows-variant2}
\end{code}

\begin{code}
\begin{minted}{java}
public class IOUtils {
	...
    /**
     * Compare the contents of two Readers to determine if they are equal or
     * not, ignoring EOL characters.
     * <p>
     * This method buffers the input internally using
     * <code>BufferedReader</code> if they are not already buffered.
     *
     * @param input1  the first reader
     * @param input2  the second reader
     * @return true if the content of the readers are equal (ignoring EOL differences), 
     *	false otherwise
     * @throws NullPointerException if either input is null
     * @throws IOException if an I/O error occurs
     * @since 2.2
     */
    public static boolean ...(Reader input1, Reader input2)
            throws IOException {
        BufferedReader br1 = toBufferedReader(input1);
        BufferedReader br2 = toBufferedReader(input2);

        String line1 = br1.readLine();
        String line2 = br2.readLine();
        while (line1 != null && line2 != null && line1.equals(line2)) {
            line1 = br1.readLine();
            line2 = br2.readLine();
        }
        return line1 == null ? line2 == null ? true : false : line1.equals(line2);
    }
    ...
}
\end{minted}
\caption{Commons IO 2.4 IOUtils.contentEqualsIgnoreEOL Variant 1}
\label{lst:ioutils.contentequalsignoreeol-variant1}
\end{code}

\begin{code}
\begin{minted}{java}
public class IOUtils {
	...
    public static boolean ...(Reader input1, Reader input2)
            throws IOException {
        BufferedReader br1 = toBufferedReader(input1);
        BufferedReader br2 = toBufferedReader(input2);

        String line1 = br1.readLine();
        String line2 = br2.readLine();
        while (line1 != null && line2 != null && line1.equals(line2)) {
            line1 = br1.readLine();
            line2 = br2.readLine();
        }
        return line1 == null ? line2 == null ? true : false : line1.equals(line2);
    }
    ...
}
\end{minted}
\caption{Commons IO 2.4 IOUtils.contentEqualsIgnoreEOL Variant 2}
\label{lst:ioutils.contentequalsignoreeol-variant1}
\end{code}

\newpage
\begin{code}
\begin{minted}{java}
public class Monitor {
/**
   * Enters this monitor. Blocks at most the given time.
   *
   * @return whether the monitor was entered
   */
  public boolean _(long time, TimeUnit unit) {
    long timeoutNanos = unit.toNanos(time);
    final ReentrantLock lock = this.lock;
    if (!fair && lock.tryLock()) {
      return true;
    }
    long deadline = System.nanoTime() + timeoutNanos;
    boolean interrupted = Thread.interrupted();
    try {
      while (true) {
        try {
          return lock.tryLock(timeoutNanos, TimeUnit.NANOSECONDS);
        } catch (InterruptedException interrupt) {
          interrupted = true;
          timeoutNanos = deadline - System.nanoTime();
        }
      }
    } finally {
      if (interrupted) {
        Thread.currentThread().interrupt();
      }
    }
  }
}

// Usage example:
monitor._(5, TimeUnit.SECONDS);
try {
	// do things while occupying the monitor
} finally {
	monitor.leave();
}
\end{minted}
\caption{Guava 17.0 Monitor.enter Variant 1}
\label{lst:guava-monitorenter-variant1}
\end{code}

\newpage
\begin{code}
\begin{minted}{java}
public class Monitor {
  public boolean _(long time, TimeUnit unit) {
    long timeoutNanos = unit.toNanos(time);
    final ReentrantLock lock = this.lock;
    if (!fair && lock.tryLock()) {
      return true;
    }
    long deadline = System.nanoTime() + timeoutNanos;
    boolean interrupted = Thread.interrupted();
    try {
      while (true) {
        try {
          return lock.tryLock(timeoutNanos, TimeUnit.NANOSECONDS);
        } catch (InterruptedException interrupt) {
          interrupted = true;
          timeoutNanos = deadline - System.nanoTime();
        }
      }
    } finally {
      if (interrupted) {
        Thread.currentThread().interrupt();
      }
    }
  }
}
\end{minted}
\caption{Guava 17.0 Monitor.enter Variant 2}
\label{lst:guava-monitorenter-variant2}
\end{code}

\subsection{Questions to ask during the interview}

\subsubsection{General questions before the interview}

\begin{enumerate}
\item Do you have practical experience with the Java programming language?
\item If yes, how many years of experience do you have?
\item With which other languages do you have practical experience?
\end{enumerate}

\subsubsection{Questions to ask per code sample}
\begin{enumerate}
\item Do you have experience with \textbf{**domain of code sample**}
\item Do you understand and can you explain what this code does?
\end{enumerate}

\section{Interview script}

\begin{enumerate}
\item Please provide a name for the method with the missing name.

\item When unable 
	\begin{enumerate}
	\item ask: Do you have experience/are familiar with \textbf{**domain of code sample**}
	\item When negative: explain/provide reading material about the domain.
	\item Please provide a name for the method with the missing name.
	\item When unable, ask: Do you understand/can you explain what this code does?
	\item When negative: give more contextual information
	\item Repeat last two steps until all contextual information is provided.
	\end{enumerate}
\item When able
	\begin{enumerate}
	\item ask: Can you explain why you chose this particular name and how did you construct it?
	\item ask: Do you have experience with \textbf{**domain of code sample**}
	\item Do you understand and can you explain what this code does?
	\end{enumerate}
\end{enumerate}
\section{Frequency of Java Dependencies}
\label{appendix:frequency-of-java-dependencies}

{\small This is an edited version of my original article as posted on my personal website: http://www.jeroenpeeters.nl/articles/frequency-of-java-dependencies/}.

\subsection{The Approach}

Github exposes an API through which you can search for projects with a certain language, rating, etc. Furthermore it is possible to query a project’s tree structure and obtain file data.

Because I needed a representative data set I choose to include mature and active projects only. For this purpose a project is considered active if it had at least one commit in the last year. Secondly a project is considered mature if it is older than at least one year.

The Java world has three major build and dependency management tools; Maven, Gradle and Ivy. I simply downloaded the according build files to obtain dependency related information.

For each Github project each dependency is only counted once. This means that if a project contains multiple modules each having its own dependency management, duplicated dependencies are counted as one occurrence. Furthermore, build tools specific dependencies (such as maven-compiler-plugin) are omitted from the results. This is because depending on these artifacts is a consequence of using the build tool. They would thus occur frequently and cloud the results.

\subsection{The Results}

From the years 2008 to 2012 I was able to retrieve 3.029 projects with dependency management files (of which 2502 (82\%) Maven projects, 430 (14\%) Gradle projects and 97 (3\%) Ant+Ivy projects).

From these figures it is not difficult to conclude that the majority of Java projects use Maven as a build and dependency management tool.

These projects had a total of 26.235 unique dependencies. The following list details the top 5 projects on which others depend:

\begin{itemize}
	\item junit - 1883
	\item slf4j-api - 764
	\item oss-parent - 700
	\item log4j - 671
	\item commons-io - 543
\end{itemize}

The following graphic shows the top 25 most depended on projects. We observe that JUnit is by far the most depended on artifact, followed by sfl4j-api and oss-parent.

We can see that a large portion of these top projects are related to testing (junit, mockito-all, spring-test, mockito-core) and logging (sfl4j-api, log4j, slf4j-log4j12, commong-logging, logback-classic).

\subsection{The code}

To obtain and analyze the Github data I had to implement two relatively small programs. Both of them are freely available under the GNU GPL from, off course, Github.

\begin{itemize}
	\item https://github.com/jeroenpeeters/github-dependency-analyzer
\end{itemize}

\subsection{The data}

The full data set contains 25.243 projects. Table \ref{table:dependency-frequency} lists the first 30 projects from the data set with their dependency frequency.
The complete data set, in raw and analyzed form, can be downloaded from the above mentioned Github repository. 

\begin{table}
	\centering
	\begin{tabular}{ | l | l | }
	    \hline
	    \textbf{Project name} & \textbf{Dependency count} \\ \hline
	    junit & 1883 \\ \hline
	    slf4j-api & 764 \\ \hline	
	    oss-parent & 700 \\ \hline
	    log4j & 671 \\ \hline
	    commons-io & 543 \\ \hline
	    guava & 520 \\ \hline
	    servlet-api & 489 \\ \hline
	    slf4j-log4j12 & 482 \\ \hline
	    commons-lang & 444 \\ \hline
	    commons-logging & 436 \\ \hline
	    mockito-all & 385 \\ \hline
	    commons-codec & 335 \\ \hline
	    lifecycle-mapping & 333 \\ \hline
	    spring-context & 332 \\ \hline
	    httpclient & 290 \\ \hline
	    spring-test & 288 \\ \hline
	    logback-classic & 284 \\ \hline
	    joda-time & 283 \\ \hline
	    jackson-mapper-asl & 273 \\ \hline
	    jcl-over-slf4j & 264 \\ \hline
	    testng & 263 \\ \hline
	    spring-core & 263 \\ \hline
	    mockito-core & 261 \\ \hline
	    android & 244 \\ \hline
	    spring-beans & 235 \\ \hline
	    spring-web & 234 \\ \hline
	    commons-collections & 228 \\ \hline
	    hsqldb & 218 \\ \hline
	    spring-webmvc & 216 \\ \hline
	    mysql-connector-java & 215 \\ \hline
	\end{tabular}
	\caption{Dependency frequency: Thirty most depended-on projects}
	\label{table:dependency-frequency}
\end{table}

%\section{The First Appendix}
%The appendix fragment is used only once. Subsequent appendices can be created
%using the Section Section/Body Tag.

\end{document}
