\section{Introduction}

\subsection{What's in a name}

Naming pieces of code is one of the most important tasks of a software developer. Giving a piece of code the correct name is important for understandability. But what is the right name? Providing good quality names is generally believed to be of high importance. Though there may be some guidelines, no exact or mechanical process exists to name pieces of code. This thesis seeks to discover the strategies people use to create names for existing pieces of code written in the Java programming language.

\subsubsection{Names in the Java programming language}


Before we continue we should understand which nameable parts exist the Java programming language.
In the Java programming language the following constructs are nameable.\cite{javaLangSpec}

\begin{list}{-}{}
\item Classes
\item Methods
\item Variables
\item Method arguments
\item Return types
\end{list}

Examples of the above constructs are listed in appendix \ref{appendix:java-nameabe-examples}.

\subsection{What's in a good name?}

For computers to interpret and execute program code there is no need for it to be laid out or indented. Just the same the computer doesn't care about nice, well understandable and self-explanatory names. We do this to communicate the intent of the code to other developers or even our future self should we come back to extend the code or fix bugs.

\subsubsection{Name characteristics}
\label{sec:name-characteristics}

\begin{list}{-}{Characteristics of names in programming:}
\item Reveal intentions
\item Length
\item One word per concept
\item Use names from the domain
\end{list}


\subsection{Characteristics in code}




Zoals besproken op Skype:

De huidige opzet van mijn scriptie/onderzoek blijkt lastig te voltooien. Jij stelde het volgende voor; ik kom een week naar de UvA om studenten en misschien docenten te interviewen over de vraag hoe mensen komen tot de naamgeving van een methode, gegeven de implementatie en context. Deze kennis kan interessant zijn om code automatisch te labelen bij automatische refactoring. Jij merkte op dat het wellicht interessant is om uit te zoeken welke karakteristieken in de code nu echt belangrijk zijn voor het geven van een naam. Het idee is om code met veel verschillende karakteristieken te geven, een complex vraagstuk. Volgens Kahnemen zal dit leiden tot een substituut vraag, welke karakteristieken zijn nu nog belangrijk? Altijd dezelfde? Of altijd de eerste N die men tegenkomt bij het lezen van de code. Maakt de context uberhaupt uit? of is alleen de implementatie genoeg? of is misschien juist alleen de context genoeg voor het geven van een naam? Worden andere namen gegeven bij afwezigheid van context of implementatie. Verder kan nog worden gekeken naar de strategie die mensen kiezen: beantwoorden ze elke vraag op eenzelfde manier waardoor de gegeven namen ook vergelijkbaar zijn opgebouwd. Mogelijkheid is ook dat mensen geen naam kunnen geven. 

Jij noemde ook inconsistentie als mogelijkheid om naar te kijken en je noemde als voorbeeld het gebruik van naamgevingsconventies; zou je nog kort kunnen toelichten wat je hiermee bedoelde?, ik kon dit uit m'n aantekeningen niet direct meer opmaken :S

Het idee is, volgens mij, om het volgende te organiseren: over 3-4 weken een week op de Uva voor het afnemen van interviews. Ik bereid vragen voor op papier met random vragen uit een codebase van github van meestgebruikte open-source software. Interview is interactief volgens 'thinking aloud protocol'.

Deze week + weekend zal ik dit plan verder uitwerken + voorbeeld vragen voorbereiden die wij maandag 9 november kunnen bespreken. Ik zal dan in de middag, na de lunch (rond 13:00u?), staan bij C302.4.

Hans, hartelijk dank voor je tijd zover en je aanbieding om studenten te interviewen.

Groet,
Jeroen