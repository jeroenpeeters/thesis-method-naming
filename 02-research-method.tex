\section{Research Method}

\subsection{Semi structured interviews}
The goal is to obtain data about how people reverse engineer method names from code. To try to answer this question I've held interviewing sessions with respondents from the field of software engineering. The group of respondents exists of students from the Software Engineering Master at the University of Amsterdam and professional software engineers from ICTU\footnote{ICTU (http://www.ictu.nl/), semi governmental agency for IT projects}. For research purposes generally two types of interviews are employed, structured and unstructured interviews.

In structured interview questions are asked from a standardized set of which the order is known. Generally more closed questions are used which makes the interview easier to replicate because they generate quantitative data. On the other hand, it is not possible to ask about motives or reason behind an answer.

An unstructured or informal interview on the other hand acts as a conversation guide. Generally more open-ended questions are used which give the opportunity to tailor the follow-up questions to the answers of the respondent. It gives the interviewer the possibility to validate answers, get deeper understanding and ask for the rationale behind an answer. This type of research is harder to replicate as it generates more qualitative data.

In my interviews I use a predefined list of code samples and questions. I will use open questions to get the conversation started. The idea is that an open question will trigger the respondent to think deeper about the code sample. While engaging in a conversation, I use a list of predefined closed questions to test for validity and ask for an explanation and rationale.

\subsection{Thinking-aloud protocol}
The thinking-aloud protocol is based on protocol analysis as described by Ericsson and Simon \citationNeeded.

\subsection{Varying code characteristics}
I've created created three variants of interviews that use the same code samples with varying degrees of contextual information. In ordre to discover which pieces of information are of most importance I will vary:

\begin{itemize}
\item JavaDoc of a method implementation
\item Method call example
\item .....
\end{itemize}

... TBD; section about how I vary context for the code samples.