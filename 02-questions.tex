\section{Research Question}

\subsection{Can people reverse-engineer method names?}
Given a method implementation and contextual information, can people reconstruct the intent of that method and give it an appropriate name? What is the minimal required amount and type of information?

\subsection{What are the strategies people employ to infer a method name?}
When people are given the task to create an appropriate names for a given method implementation and its context, what strategies do they follow? 

\subsection{Hypothesis}

The main driver of the hypotheses is the intent of the code. The intent of the code primarily exhibits in the names {\color{red}[CITATION NEEDED]}. Therefore the hypotheses are grouped per characteristic (\ref{sec:name-characteristics}).
%The main driver behind the hypotheses are the name characteristics  (see %\ref{sec:name-characteristics}). The hypotheses are therefore grouped per characteristic.

%Hoe kom je tot hypothese: systematisch elementen van code. intentie als de primarre driver. intentie zit in naamgeving.

\subsubsection{Reveal intentions}
\begin{subhyp}
	\begin{hyp}
 		When the code is simple and uses coherent names, comments will be ignored.
	\end{hyp}
	\begin{hyp}
 		Smaller methods, such as one liners, are missing context. Without this context it will be impossible to get the intent and thus provide an appropriate name.
	\end{hyp}
\end{subhyp}


\subsubsection{Length}
\begin{subhyp}
	\begin{hyp}
		The size of the method is positively related to the length of the method name. People will give larger names to methods with a bigger size.
	\end{hyp}
\end{subhyp}

\subsubsection{Complexity}
\begin{subhyp}
	\begin{hyp}
		The complexity of a method is negatively related to ability of people to give a method a sensible name. The more complex a method, the harder to name the method, if possible at all. 
	\end{hyp}
\end{subhyp}

\subsubsection{One word per concept}
\begin{subhyp}
	\begin{hyp}
		When the code exhibits obvious and coherent names these names will be re-used in the method name.
	\end{hyp}
\end{subhyp}

\subsubsection{Use names from the domain}
\begin{subhyp}
	\begin{hyp}
		People without domain knowledge will not be able to grasp the intent of the code. They will rely more on context, such as comments, than what the code actually expresses.
	\end{hyp}
	\begin{hyp}
		People without domain knowledge will have more difficulties naming larger and complex methods.
	\end{hyp}
\end{subhyp}


%\item When given large samples of code, people mostly scan instead of reading thoroughly. Direct opvallende dingen. Sterke aanwijzing -> vuistregels. als er 1 duidelijke naam in staat zal die naam terugkomen in de naam. repeterende naam.
%\item When people are familiar with the domain they are able to provide better names.
%Mensen zonder kennis van het domein kunnen niet goed te intentie van de code achterhalen.
%meer afgaan op wat ipv waarom. 2. meer aandacht aan commentaar -> commentaar belangrijker. Meer moeite met grote methodes.

%\item Grote methods moeilijk naam te geven.


%\item When the intent of the code is not expressed in the code itself, it will be nearly impossible to provide a suitable name.
%\item vraag: in hoeverre komt naam van methode.. intentie van methode komt tot uidrukki ng in goede naamgeving van variabele en namen. Wanneer er een coherente naamgeving is, zal deze gebruikt worden in de naam.


%\item Code comments will be ignored, even though they express the intent of the code.
%\item Without providing context, simple code samples such as one liners, will be hard to name.
\subsubsection{Patterns}
Program code exhibiting common patterns is generally easier to understand when the developer is familiar with these patterns. {\color{red}[CITATION NEEDED]}.
\begin{subhyp}
	\begin{hyp}
		Names given to pieces of code obtained by Extract-Method Refactoring are of better quality because it is a process the developer is familiar with.
	\end{hyp}
	\begin{hyp}
		When the code exhibits common patterns with which the developer is familiar, he will answer the question from experience. \newline [BETER VERWOORDEN]
	\end{hyp}
\end{subhyp}


\subsection{Hypothesis Parameter Variability Matrix}

\begin{tabular}{ *5l }
\emph{Hypothesis} & \multicolumn{4}{l}{\emph{Parameter variation}} \\
\hline
   & Javadoc  & Call example  & Size  & Complexity \\ 
H1 & X1 & X2 & X3 & X4 \\ 
H2 & Y1 & Y2 & Y3 & Y4 \\
H3 & Y1 & Y2 & Y3 & Y4 \\
H4 & Y1 & Y2 & Y3 & Y4 \\
H5 & Y1 & Y2 & Y3 & Y4 \\
H6 & Y1 & Y2 & Y3 & Y4 \\
\hline
\end{tabular}

Matrix met parameters

toon classes, documentatie, grote methode. welke parameters ga ik varieren. mensen die commentaar niet gebruiken, ervaring met onbruikbaar commentaar.

Frank Nack. Sander Bakkes. -> UXlab